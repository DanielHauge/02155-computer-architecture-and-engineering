\section*{A1.7}
\subsection*{a.}
If RISC-V only had 2 argument registers and one result register, I would store the arguments in memory and pass the address of the arguments rather than the actual values.
To return the results, again the results could be stored in memory and address to the results would be given instead of actual values. 
This way composite structures like imaginary numbers, but also other structures that contain multiple values like strings can be used.
Both techniques are fundamental when looking at procedure call patterns: \textbf{Call By Reference} and \textbf{Call By Value} that is used in high level programming languages.

\subsection*{b.}
No registers are saved across a procedure call without additional instructions.
Only registers that is put in the stack or stored elsewhere in memory with instructions will be saved.
It is convention to save and restore return address and used temporaries to x8-x9 and x18-x27 registers from the stack.