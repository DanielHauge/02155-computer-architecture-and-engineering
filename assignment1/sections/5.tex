\section*{A1.5}

\begin{figure}[H]
    \centering{\caption{Complex multiplication program}\begin{minted}[linenos,frame=single]{nasm}
        . data
aa :    . word a # Re part of z
bb :    . word b # Im part of z
cc :    . word c # Re part of w
dd :    . word d # Im part of w
        
        . text
        . globl main
main :
    lw a0, aa
    lw a1, bb
    lw a2, cc
    lw a3, dd
    jal ra, complexMul # Multiply z and w
    nop
    j end # Jump to end of program
    nop

complexMul:
    MUL t0, a0, a2
    MUL t1, a1, a3
    MUL t2, a0, a3
    MUL t3, a1, a2
    SUB a0, t0, T1
    ADD a1, t2, t3
    jalr x0, 0(ra)

end:
    nop
    \end{minted}
    }
\end{figure}

\subsection*{a.}
The procedure \textit{complexMul} is a leaf procedure, as it does not call any other procedures.
\subsection*{b.}
The stack was not used, as complex multiplication was achievable with reasonable convinience without the need to retain local variables that a non-leaf procedure needs.
For example, no need to store return address or other temporary values that will be needed after secondary procedure call returns.

\subsection*{c.}
4 MUL instructions contribute 8 cycles, sub, add and jalr contribute 3 cycles, which totals 11 cycles.