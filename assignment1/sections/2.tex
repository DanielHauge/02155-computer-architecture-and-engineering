\section*{A1.2}

\begin{figure}[H]
    \centering{\caption{Vector reader program}\begin{minted}[linenos,frame=single]{nasm}
ADDI t0, x0, 0x10000    // Using t0 as read address for number in vector
ADD t1, x0, x0          // Using t1 as index i starting at 0
ADDI t2, x0, 100        // Using t2 for stop condition
ADDI a0, x0, 0x8000     // Init smalest number as largest value  32767
ADDI a1, x0, 0x7FFF     // Init largest number as smalest value -32768

LOOP:
BEG t1, t2, END         // End when i reach 100, meaning all numbers are read
LW t3, 0(t0)            // Load number from vector

BGE t3, a1, BIGGER      // Branch if number is largest so far
MaybeSmaller:       
BLT t3, a0, SMALLER     // Branch if number is smalest so far
BNE x0, x0, CONTINUE   

BIGGER:                 
ADDI a1, x0, t3         // Set largest number seen so far
BNE x0, x0, MaybeSmaller

SMALLER: 
ADDI a0, x0, t3         // Set smalest number seen so far

CONTINUE:
ADD t1, t1, 1           // Increment i by one (i++)
ADDI t0, t0, 4          // Increment address to next number (4 bytes)
BNE x0, x0, LOOP        // Always branch to start of loop.

END:
NOP
    \end{minted}
    }
\end{figure}

I considered the case where the hundred numbers is always incrementing when reading the numbers sequentialy.
To accomodate for this, going back with "MaybeSmaller" is used to also check if number is smaller than what has been observed.

\newpage