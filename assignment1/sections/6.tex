\section*{A1.6}

\begin{figure}[H]
    \centering{\caption{Alternative Ccmplex multiplication program}\begin{minted}[linenos,frame=single]{nasm}
        . data
aa :    . word a # Re part of z
bb :    . word b # Im part of z
cc :    . word c # Re part of w
dd :    . word d # Im part of w
        
        . text
        . globl main
main :
    lw a0, aa
    lw a1, bb
    lw a2, cc
    lw a3, dd
    jal ra, altComplexMul # Multiply z and w
    nop
    j end # Jump to end of program
    nop

altComplexMul:
    ADD t0, a0, a1
    MUL t0, t0, a2
    SUB t1, a3, a2
    MUL t1, t1, a0
    ADD t2, a2, a3
    MUL t2, t2, a1 
    SUB a0, t0, t2
    ADD a1, t0, t1
    jalr x0, 0(ra)

end:
    nop
    \end{minted}
    }
\end{figure}

\subsection*{a.}
\textit{altComplexMul} is a leaf procedure as it does not call another procedure.
\subsection*{b.}
The stack is not used as there was no need to save return address, temporaries or any values from registers that might be overwritten in a nested procedure call.

\subsection*{c.}
3 MUL instructions contribute 6 cycles, 3 ADD instructions contribute 3 cycles, 2 SUB instructions contribute 2 cycles and jalr contribute one cycle, totaling 12 cycles