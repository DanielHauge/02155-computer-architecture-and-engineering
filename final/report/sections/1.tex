\section{Introduction}

This report covers the final assignment of course 02155 Computer Architecture and Engineering.
The objective of the assignment is to simulate the base RS32I instruction set of a RISC-V processor.
The following instructions is ignored however:
\begin{itemize}
    \item fence
    \item fence.i
    \item ebreak
    \item csrrw
    \item csrrs
    \item csrrc
    \item csrrwi
    \item csrrsi
    \item csrrci
\end{itemize}

The simulator is able to ingest a binary file, that is compiled with RISC-V RS32I as target instruction set.
The ingested binary file is executed and as a result, print a correct register state (according to the program binaries).

Additionally, an output path can be specified to write the register values to little endian encoded binaries, starting from x0 to x31.

The simulator can be run on ubuntu with the following instructions:

\begin{figure}[H]
    \centering{\caption{Install golang (\href{https://go.dev/dl/go1.19.3.windows-amd64.msi}{windows installer})}
        \begin{minted}[linenos,frame=single]{bash}
$ curl -O $HOME/go.tar.gz -L https://go.dev/dl/go1.19.3.linux-amd64.tar.gz 
$ rm -rf $HOME/go && tar -C $HOME -xzf go.tar.gz
$ export PATH=$PATH:$HOME/go/bin
$ go version
go version go1.19.3 linux/amd64
        \end{minted}
    }
\end{figure}
If succesfull, go version should be printed.
The simulator can then be used in the following 2 ways, given current directory is in source code root:

\begin{figure}[H]
    \centering{\caption{Run (Build \& run)}
        \begin{minted}[linenos,frame=single]{bash}
$ go run . ./tests/binary/t1.bin ./register_dump.bin
        \end{minted}
    }
\end{figure}

\begin{figure}[H]
    \centering{\caption{Build executable}
        \begin{minted}[linenos,frame=single]{bash}
$ go build -o executable
$ ./executable ./tests/binary/t1.bin ./register_dump.bin
        \end{minted}
    }
\end{figure}